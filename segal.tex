\documentclass[./main.tex]{subfiles}
\begin{document}
 
\begin{prop}[Segal Condition]
  
  Let $\CAT$ be the (1-)category of (small) categories.
  Define the \emph{nerve functor} $N : \CAT \to \SSET$ 
  as the nerve along the forgetful functor $\bbDe \to \CAT$.

  Let $h : \SSET \to \CAT$ by sending $X$ to 
  the \emph{homotopy category of $X$}, denoted $h X$,
  defined as follows : 
  \footnote{
    Land 1.2.5.
  }
  \begin{itemize}
    \item objects of $h X$ are points of $X$
    \item For simplicity, call sequences of composable edges in $X$
    \emph{chains} in $X$.
    Given two chains $\ga, \de$ in $X$,
    we say $\ga$ is a \emph{1-step simplification of $\de$} when
    $\de$ can by obtained from $\ga$ by either : 
    \begin{itemize}
      \item replacing an edge $h$ in $\ga$ by a length two chain $f , g$
      whenever we have a triangle $\si : \De^2 \to X$ of the form : 
    \begin{cd} 
      &&&&&&&& {} \\
      \dots & \bullet & \bullet & \bullet & \dots &
        \rightsquigarrow & \dots & \bullet && \bullet & \dots \\
      &&&&&&&& \bullet
      \arrow["f"', from=2-8, to=3-9]
      \arrow["g"', from=3-9, to=2-10]
      \arrow[""{name=0, anchor=center, inner sep=0}, "h", from=2-8, to=2-10]
      \arrow[from=2-7, to=2-8]
      \arrow[from=2-10, to=2-11]
      \arrow[from=2-1, to=2-2]
      \arrow["f"', from=2-2, to=2-3]
      \arrow["g"', from=2-3, to=2-4]
      \arrow["\sigma", shorten >=3pt, Rightarrow, from=3-9, to=0] 
    \end{cd}
      \item inserting a constant edge in $\ga$
      \begin{cd}
        \dots & \bullet & \bullet & \dots & \rightsquigarrow & \dots & \bullet & \dots
        \arrow[from=1-6, to=1-7]
        \arrow[from=1-7, to=1-8]
        \arrow[from=1-1, to=1-2]
        \arrow["id", from=1-2, to=1-3]
        \arrow[from=1-3, to=1-4]
      \end{cd}
      (This is actually covered by the previous case via
      degenerate triangles.)
    \end{itemize} 
    We say $\ga$ is a \emph{simplification} of $\de$ when
    there's a finite sequence of 1-step simplifications from $\de$ to $\ga$.

    We declare $\ga \sim \de$ when there exists a chain $\ep$ that
    simplifies to both $\ga$ and $\de$.
    This defines an equivalence relation on chains in $X$
    and we define morphisms in $h X$ to be chains in $X$ up to this equivalence.

    \item Composition of morphisms is given by concatenation of chains.
    This respects simplifications and hence is well-defined on $h X$.
    \item Associativity of composition comes from associativity of
    concatenation of chains.
    \item For every point $x$ in $X$, the equivalence class of
    the constant edge $id_x$ works as the identity morphism of $x$ in $h X$.
  \end{itemize}

  Then 
  \begin{enumerate}
    \item (Segal-Condition)\footnote{Land 1.1.52.}
    a simplicial set $X$ is in the essential image of $N$ if and only if
    for all $n \in \N$, we have a bijection 
    \[
      \SSET(\De^{n+2}, X) \overset{\iso}{\to} \SSET(I^{n+2}, X)  
    \]
    where $I^{k}$ is the \emph{spine of $\De^k$},
    the sub-simplicial set of $\De^k$
    generated by the edges ``$0 \to 1, 1 \to 2, \dots, k-1 \to k$''.

    \item $N$ is fully faithful.
    Henceforth, we can see $\CAT$ as a subcategory of $\SSET$.
     
    \item Given $X \in \SSET$, we have a natural morphism $h : X \to N(h X)$
    defined by Yoneda's lemma and the following commutative square
    natural in the first argument : 
    \begin{cd}
      {\mathbf{sSet}(\Delta^n , X)} & {\mathbf{sSet}(\Delta^n , N (h X))} \\
      {\mathbf{sSet}(I^n , X)} & {\mathbf{sSet}(I^n , N(hX))}
      \arrow["\cong"', from=1-1, to=2-1]
      \arrow["\cong", from=1-2, to=2-2]
      \arrow[from=2-1, to=2-2]
      \arrow[from=1-1, to=1-2]
    \end{cd}
    where the bottom horizontal map is given by
    realising edges in $X$ as morphisms in $h X$.
    Pre-composition with $h$ gives 
    a bijection 
    \[
      \CAT(h X , C) \map{\iso}{} \SSET(X , N C)  
    \] 
    functorial in $X$ and $C$,
    defining an adjunction $h \dashv N$.
    \footnote{Land 1.2.18.}

    Hence $\CAT$ is equivalent to 
    a full reflective subcategory of $\SSET$. 
    In particular, since $\SSET$ is complete and cocomplete,
    so is $\CAT$ and its limits are computed in $\SSET$
    and colimits computed by taking the reflection
    of the colimit in $\SSET$. 
    Since every inclusion of a full reflective subcategory is monadic,
    we can think of categories as simplicial sets equipped with
    an operation to compose edges.
  \end{enumerate}

\end{prop}
\begin{proof}
  \textit{(1)} Nerves of categories satisfy the Segal condition
  because they have a composition operation.
  
  Now let $X \in \SSET$ satisfy the Segal condition.
  The idea is that the Segal condition is enough
  to recover all the higher simplicies in $X$ from $N (h X)$. 
  From the definition of $X \to N(h X)$, 
  we also see that to prove it is an isomorphism,
  it suffices for $\SSET(I^n , X) \iso \SSET(I^n , N(h X))$.
  Since the map takes a length $n$ chain in $X$ to 
  a length $n$ chain in $N(h X)$,
  it suffices to show $\SSET(I^1 , X) \iso \SSET(I^1 , N(h X))$.
  For surjectivity, note that
  any chain in $X$ has a simplification to a single edge
  by the surjectivity part of the Segal condition.

  For injectivity,
  let $p, q : \De^1 \to X$ such that $[p] = [q] : \De^1 \to N(h X)$.
  This means there exists $\ga : I^n \to X$ with $n \geq 1$
  that simpifies both to $p$ \emph{and} to $q$,
  in possibly different ways.
  However, we know that $\ga : I^n \to X$ extends
  to a $\tilde{\ga} : \De^n \to X$ by 
  the surjectivity part of the Segal condition.
  Now we inspect how $\ga$ can simplify to $p$.
  \begin{cd}
    {I^1} & \cdots & {I^{n-1}} & {I^n} \\
    &&&& X \\
    {[0 \to n]} & \cdots & {\Delta^{n-1}} & {\Delta^n}
    \arrow[from=1-4, to=1-3, rightsquigarrow]
    \arrow[from=1-3, to=2-5]
    \arrow[from=1-3, to=1-2, rightsquigarrow]
    \arrow[from=1-2, to=1-1, rightsquigarrow]
    \arrow[from=1-4, to=2-5, "{\ga}"]
    \arrow[from=1-4, to=3-4]
    \arrow[from=3-4, to=2-5, "{\tilde{\ga}}"']
    \arrow[from=3-3, to=3-4]
    \arrow[from=1-3, to=3-3]
    \arrow[from=3-2, to=3-3]
    \arrow[from=1-2, to=3-2]
    \arrow[from=3-1, to=3-2]
    \arrow[from=1-1, to=3-1]
    \arrow["p"{description, pos=0.4}, from=1-1, to=2-5]
    \arrow[from=3-1, to=2-5]
  \end{cd}
  The top row shows a sequence of 1-step simplifications from $\ga$ to $p$.
  Each step is witness by a triangle in $X$ composing two edges
  from the previous step.
  However, such triangles must in fact be 2-dimensional faces of $\tilde{\ga}$
  by the injectivity part of the Segal condition at $n = 2$.
  So we see that each step $I^k \to X$
  must in fact factor through a sub-simplex $\De^k \to \De^n$
  containing the edge $0 \to n$.
  Simplifying all the way down, we see that $p$ must be
  the restriction of $\tilde{\ga}$ to the edge $0 \to n$.
  By the same argument, this must also be $q$
  and hence $p = q$.

  \textit{(2)}
  This can be proved either by 
  showing a bijection $\CAT(C , D) \iso \SSET(N C , N D)$ functorial in $C , D$
  or showing that $h N C \iso C$ functorially in $C$.
  We will show the former.
  Land chooses to show the latter (See Land - Corollary 1.2.14).

  Each morphism $NC \to ND$
  defines a functor $C \to D$
  by following where edges go.
  Composition is preserved because of the Segal condition at $n = 2$.
  Identity morphisms are preserved because image of
  these are precisely the constant edges in the nerve and 
  image of constant edges are constant edges 
  One can recover the original morphism $NC \to ND$ from the functor $C \to D$
  via the following commutative square natural in the first argument :
  \begin{cd}
    {\mathbf{sSet}(\Delta^n , NC)} & {\mathbf{sSet}(\Delta^n , N D)} \\
    {\mathbf{sSet}(I^n , NC)} & {\mathbf{sSet}(I^n , N D)}
    \arrow["\cong"', from=1-1, to=2-1]
    \arrow["\cong", from=1-2, to=2-2]
    \arrow[from=2-1, to=2-2]
    \arrow[from=1-1, to=1-2]
  \end{cd}
  where the vertical bijections are due to the Segal condition.
  The above square also gives a way of
  extending any functor $C \to D$ to a morphism of nerves.
  This gives a bijection $\SSET(N C , N D) \iso \CAT (C , D)$
  and is easily seen to be functorial in $C, D$.

  \textit{(3)}
  It suffices to show that for every $C \in \CAT$
  the map \[
    \SSET(N(h X) , N C) \to \SSET(X , N C)   
  \]
  is bijective.

  For injectivity, let $F , G : h X \to C$ be functors 
  such that $F h = G h$.
  We can see that $F, G$ agree on objects.
  To see the same for morphisms,
  let $p$ be a morphism in $h X$.
  Then there exists $\ga : I^n \to X$ representing $p$.
  This means that in $h X$, $h \ga$ composes to $p$.
  Then $F p = F h \ga = G h \ga = G p$.

  For surjectivity, let $F : X \to N C$.
  Define a functor $\tilde{F} : h X \to C$ as follows : 
  \begin{itemize}
    \item on objects, do what $F$ does to points.
    \item Map chains in $X$ to morphisms in $C$ by taking image under $F$ 
    then using the composition operation of $C$.
    Simplifications $\de \rightsquigarrow \ga$ are mapped to
    the same morphism in $C$ by associativity of composition in $C$,
    and hence gives a way of mapping morphisms in $h X$ to morphisms in $C$.
    \item Compositions are preserved by the Segal condition on $N C$.
    Identity morphisms are preserved since constant paths in $C$
    are exactly identity morphisms.
  \end{itemize}
  Viewing $\CAT$ as a subcategory of $\SSET$,
  we see that we need to show $\tilde{F} h = F$.
  By the density theorem,
  it suffices to check for each $x : \De^n \to X$
  that $\tilde{F} h x = F x$.
  But by the Segal condition, 
  it suffices to check $\tilde{F} h x = F x$ on the spine $I^n$.
  The spine is determined by its $n$ edges,
  so it suffices to check $\tilde{F} h x = F x$ for $x : \De^1 \to X$.
  But this is true since $x$ represents $h x$. 

\end{proof}

\end{document}