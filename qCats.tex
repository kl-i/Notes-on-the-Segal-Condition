\documentclass[./main.tex]{subfiles}
\begin{document}

\begin{prop}[Better Description of Homotopy Category
  \footnote{
    This is Land 1.2.9, 1.2.10, 1.2.11 combined.
  }]
   
  Let $X \in \SSET$. 
  Suppose $X$ lifts against inner 2-horns and inner 3-horns.
  We can then define a category $\pi X$ as follows : 
  \begin{itemize}
    \item objects of $pi X$ are points of $X$.
    \item
    For $x , y : \De^0 \to X$,
    the following two relations on $X(x,y)$ are the same : 
    \begin{itemize}
      \item $f \sim g$ when there exists 
      a triangle in $X$ of the following form : 
      \begin{cd}
        \bullet & \bullet \\
        & \bullet
        \arrow[""{name=0, anchor=center, inner sep=0}, "g"{description}, from=1-1, to=2-2]
        \arrow["f"{description}, from=1-1, to=1-2]
        \arrow["id"{description}, from=1-2, to=2-2]
        \arrow[shorten >=2pt, Rightarrow, from=1-2, to=0]
      \end{cd}
      \item $f \sim g$ when there exists
      a triangle in $X$ of the following form : 
      \begin{cd}
        \bullet & \bullet \\
        \bullet
        \arrow["f"{description}, from=1-1, to=1-2]
        \arrow[""{name=0, anchor=center, inner sep=0}, "g"{description}, from=2-1, to=1-2]
        \arrow["id"{description}, from=1-1, to=2-1]
        \arrow[shorten >=2pt, Rightarrow, from=1-1, to=0]  
      \end{cd}
    \end{itemize}
    In fact, this defines an equivalence relation.
    Define $\pi X(x, y)$ to be the quotient of $X(x,y)$ by it.
    \item For $I^2 \to X$ representing two composable edges $f , g$,
    define \emph{composites} of $f , g$ to be
    extensions of $I^2 \to X$ to all of $\De^2$.
    Then for any pair of composable edges in $X$,
    a composite exists and is unique up to the equivalence,
    hence defining a composition operation for $\pi X$. 
  \end{itemize}
  Then $h X \iso \pi X$.
\end{prop}
\begin{proof}
  
  \textit{(The two relations)}
  \begin{cd}
    &&&&& \bullet & \\
    \bullet & \bullet & \Rightarrow & \bullet & \bullet && \Lambda^3_1 \\
    \bullet &&& \bullet && \bullet &&\\
    \bullet & \bullet & \Rightarrow && \bullet & \bullet & \Lambda^3_2 \\
    & \bullet && \bullet &&
    \arrow["f"{description}, from=2-1, to=2-2]
    \arrow[""{name=0, anchor=center, inner sep=0}, "g"{description}, from=3-1, to=2-2]
    \arrow["id"{description}, from=2-1, to=3-1]
    \arrow["f"{description}, from=2-5, to=3-6]
    \arrow["g"{description}, from=2-4, to=3-6]
    \arrow["f"{description}, from=2-4, to=1-6]
    \arrow["id"{description}, from=1-6, to=3-6]
    \arrow["f"{description}, from=4-1, to=4-2]
    \arrow[""{name=1, anchor=center, inner sep=0}, "g"{description}, from=4-1, to=5-2]
    \arrow[from=4-2, to=5-2]
    \arrow["id"{description}, from=2-4, to=2-5]
    \arrow["f"{description}, from=2-5, to=1-6]
    \arrow["f"{description}, from=3-4, to=4-6]
    \arrow["f"{description}, from=5-4, to=4-5]
    \arrow["f"{description}, from=3-4, to=4-5]
    \arrow["id"{description}, from=5-4, to=3-4]
    \arrow["g"{description}, from=5-4, to=4-6]
    \arrow["id"{description}, from=4-5, to=4-6]
    \arrow[shorten >=2pt, Rightarrow, from=2-1, to=0]
    \arrow[shorten >=2pt, Rightarrow, from=4-2, to=1]
  \end{cd}

  \textit{(Equivalence Relation)}
  Symmetry and transitivity are given respectively by : 
  \begin{cd}
    &&&&& \bullet \\
    \bullet & \bullet & \Rightarrow & \bullet & \bullet && {\Lambda^3_1} \\
    & \bullet && \bullet && \bullet \\
    \bullet & \bullet & \Rightarrow && \bullet & \bullet & {\Lambda^3_2} \\
    & \bullet && \bullet
    \arrow[""{name=0, anchor=center, inner sep=0}, "g"{description}, from=2-1, to=3-2]
    \arrow["f"{description}, from=2-1, to=2-2]
    \arrow["id"{description}, from=2-2, to=3-2]
    \arrow["f"{description}, from=2-4, to=2-5]
    \arrow["id"{description}, from=2-5, to=1-6]
    \arrow["g"{description}, from=2-4, to=3-6]
    \arrow["id"{description}, from=2-5, to=3-6]
    \arrow["id"{description}, from=3-6, to=1-6]
    \arrow["f"{description}, from=2-4, to=1-6]
    \arrow["g"{description}, from=4-1, to=4-2]
    \arrow[""{name=1, anchor=center, inner sep=0}, "h"{description}, from=4-1, to=5-2]
    \arrow["id"{description}, from=4-2, to=5-2]
    \arrow["f"{description}, from=5-4, to=3-4]
    \arrow["id"{description}, from=3-4, to=4-5]
    \arrow["id"{description}, from=4-5, to=4-6]
    \arrow["g"{description}, from=5-4, to=4-5]
    \arrow["h"{description}, from=5-4, to=4-6]
    \arrow["id"{description}, from=3-4, to=4-6]
    \arrow[shorten >=2pt, Rightarrow, from=2-2, to=0]
    \arrow[shorten >=2pt, Rightarrow, from=4-2, to=1]
  \end{cd}

  \textit{(Composites)}
  Existence of composites in $X$ is precisely
  the fact that $X$ lifts against the inner 2-horn.
  For uniqueness of composites up to equivalence, use : 
  \begin{cd}
    && \bullet \\
    \bullet & \bullet && {\Lambda^3_1} \\
    && \bullet
    \arrow["f"{description}, from=2-1, to=2-2]
    \arrow["g"{description}, from=2-2, to=1-3]
    \arrow["g"{description}, from=2-2, to=3-3]
    \arrow["id"{description}, from=1-3, to=3-3]
    \arrow["{h'}"{description}, from=2-1, to=3-3]
    \arrow["h"{description}, from=2-1, to=1-3]
  \end{cd}

  \textit{(Associativity of Composition)}
  Let $f , g , h$ be a triple of composable edges in $X$.
  Then we have 
  \begin{cd}
    && \bullet \\
    \bullet & \bullet && {\Lambda^3_1} \\
    && \bullet
    \arrow["f"{description}, from=2-1, to=2-2]
    \arrow["g"{description}, from=2-2, to=1-3]
    \arrow["hg"{description}, from=2-2, to=3-3]
    \arrow["h"{description}, from=1-3, to=3-3]
    \arrow["{(hg)f}"{description}, from=2-1, to=3-3]
    \arrow["{g f}"{description}, from=2-1, to=1-3]
  \end{cd}
  where $gf$ is a composite of $f , g$ and $h g$ is
  a composite of $g , h$,
  and $(hg) f$ is a composite of $hg$ with $f$. 
  Filling the above inner horn shows that
  $(hg) f$ is in fact also a composite for $h$ with $g f$.
  By uniqueness of composites up to equivalence,
  this proves associativity of composition.

  $(h X \iso \pi X)$
  From the universal property of $h : X \to N(h X)$,
  we have a commuting triangle of simplicial sets : 
  \begin{cd}
    X & \\
    N(h X) & N(\pi X)
    \arrow[from=1-1, to=2-1, "{h}"']
    \arrow[from=2-1, to=2-2]
    \arrow[from=1-1, to=2-2]
  \end{cd}
  By fully faithfulness of $N$,
  we need to show the horizontal morphism is an isomorphism.
  It is bijective at the level of objects.
  So we look at the level of morphisms.
  Let $x, y$ be points of $X$.
  Then 
  \begin{cd}
    \SSET(\De^1 , X) & \\
    {hX(x,y)} & {\pi X(x ,y)}
    \arrow[from=1-1, to=2-1, "{h}"']
    \arrow[from=2-1, to=2-2]
    \arrow[from=1-1, to=2-2]
  \end{cd}
  The diagonal map is surjective by definition,
  hence so is the bottom map.
  The vertical map is also surjective : 
  1-step simplifications are obtained by composites of pairs of morphisms.
  So to show injectivity of the bottom map,
  we can take two $f , g \in h X(x , y)$,
  lift them to edges in $X$ then quotient to $\pi X (x, y)$.
  If they are equal in $\pi X( x , y)$,
  then there is a triangle in $X$ witnessing their equivalence.
  The spine of this triangle simplifies both to $f$ and $g$ as
  edges in $X$,
  and thus $f = g \in h X(x,y)$ as desired.
\end{proof}

\begin{intuit}

  Let $X \in \SSET$ and $(f , g) : \La^2_1 \to X$ a pair of
  composable edges.
  Then the ``space'' of composites of $f$ with $g$ can be
  defined by the following cartesian square : 
  \begin{cd}
    {\mathrm{Comp}(f,g)} & {X^{\Delta^2}} \\
    {\Delta^0} & {X^{\Lambda^2_1}}
    \arrow["{(f,g)}"', from=2-1, to=2-2]
    \arrow[from=1-1, to=2-1]
    \arrow[from=1-1, to=1-2]
    \arrow[from=1-2, to=2-2]
    \arrow["\lrcorner"{anchor=center, pos=0.125}, draw=none, from=1-1, to=2-2]
  \end{cd}  
  In the above proposition, we saw that being able to lift against 
  inner 3-horns implies that $\pi_0^\De \mathrm{Comp}(f,g)$ is singleton.
  The incarnation of infinity categories as \emph{quasi-categories}
  will say that the family $X^{\De^2} \to X^{\La^2_1}$ 
  is ``homotopically trivial''.
  In particular,
  we will much later show that 
  $\mathrm{Comp}(f,g)$ is a contractible Kan complex.  
\end{intuit}

\begin{prop}[Equivalent Definitions of Infinity Categories (Preview)
  \footnote{
    Land - 1.3.34
  }]
  Let $X$ be a simplicial set.
  Define the following classes of morphisms of simplicial sets : 
  \begin{itemize}
    \item inner horns $\IH = \set{\La^n_k \to \De^n \st n \geq 2 ,
    0 < k < n}$.
    \item cell inclusions $\CELL = \set{\partial\De^n \to \De^n \st n \geq 0}$
  \end{itemize}
  Then the following are equivalent : 
  \begin{itemize}
    \item $X$ lifts against $\IH$
    \item (Homotopy Unique Composition) 
    $X^{\De^2} \to X^{\La^2_1}$ lifts against $\CELL$.
    \item (Homotopical Segal Condition)
    For $n \geq 2$,
    $X^{\De^n} \to X^{I^n}$ lifts against $\CELL$. 
  \end{itemize}
  Call $X$ an \emph{infinity category} when
  it satisfies any (and thus all) of the above.
\end{prop}

\begin{eg}
  \begin{itemize}
    \item Infinity groupoids $S(X)$ of topological spaces $X$, 
    i.e. nerves of topological spaces along 
    the standard realisation of simplicies $\bbDe \to \TOP$.
    \item Nerves of 1-categories.
    \item Later we will be able to turn categories
    enriched over $\Z$-complexes into infinity categories
    via a procedure called the \emph{dg-nerve}.
    
    This will give examples from homological algebra and algebraic geometry : 
    the dg-nerve $N_{dg} (\mathrm{Ch} A)$ of (unbounded) complexes
    in a (Grothendieck) abelian category.
    \item In order to take ``higher'' (co)limits of infinity categories,
    we will have to make an infinity category $\CAT_{\infty}$ 
    of (small) infinity categories.
    More important yet : 
    we will want to make ``families of infinity categories over 
    a base infinity category $B$'' which vary contravariantly along $B$.
    These should correspond to functors $F : B\op \to \CAT_\infty$.
    This is the subject of Lurie's straightening-unstraightening theorem.

  \end{itemize}
\end{eg}
\end{document}